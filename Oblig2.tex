Chapter 7

Theory Questions

1. Hva brukes en bitmap til i forbindelse med minnehåndtering i et page-basert minne?

Bitmap benyttes for å holde rede på ledig plass, gjerne i RAM eller på disk. Ettersom bitmap ikke trenger så stor plass for å adressere store minnemengder er det relativt effektivt.

2. Hvilke felter finnes i en Pagetable Entry ?

En pagetable entry er delt opp i flere felt, present/absent bit, modified bit, referenced bit, caching disabled bit, den inneholder også pekeren mellom det virtuelle minnet og det fysiske minnet. Muligens også adresse rom eller PID til eier.
Se http://en.wikipedia.org/w/index.php?title=Page_table&oldid=641651423#Page_table_data

3.  Hvilke prinsipper kan vi benytte for å redusere størrelsen på en pagetable i internminnet?

Vi har to prinsipper for å redusere størrelsen på en pagetabel, for det første kan vi dele den opp i flere lag - "Multilevel page table" - noe som gir oss mulighet till å kun holde på pages til de pagene som er i bruk nå og kan også la oss skalere opp tabellene ved behov. Ett annet prinsipp er "Inverted page table", hvor man har page frames adressert i stedet for en entry i en page fram adressert. Dette fører til økt kompleksitet for å gå fra virtuellet til fysisk minne. For å løse det har man en TLB for å huske de mest brukte.

 4. Affinity scheduling (“CPU pinning”) reduserer antall cache misser. Reduseres også antall TLB misser? Reduseres også antall page faults? Begrunn svaret.

Antall TLB misses reduseres ikke av CPU pinning ettersom det kun er en TLB i maskinen. Dette gjør også at page faults ikke blir redusert.
 
http://pages.cs.wisc.edu/~remzi/OSTEP/vm-tlbs.pdf

5. Hvor stor blir en bitmap i et page-inndelt minnesystem med pagestørrelse 4KB og internminne på 512MB?

512MB = 2^28 
4KB   = 2^12 

Antall 4KB pages i 512MB minne = 2^28 / 2^12 = 65 536
1 page krever 1 bit = 65 536 bits = 2^16 

Bitmappet blir 2^16 stort.

6. Tanenbaum oppgave 3.5
For each of the following decimal virtual addresses, compute the virtual page number and offset for a 4K page and for an 8K page: 20000, 32768, 60000. 

Virtual Page number ved 4K pages starter på page 0 mellom 0 - 4095 og er 0 + N for hver multiple av 4096 oppover. Vi kan dermed finne VPN for den desimale representasjonen av den virtuelle adressen ved å bruke heltalls deling på 4096. 
20000/4096 = 4
4*4096 = 16384 
20000 - 16384 = 3616
Adresse 20 000 ligger i virtual page 4 med en offset på 3616

32768/4096 = 8
8*4096 = 32768
32768-32768 = 0
Adresse 32768 ligger i virtual page 8 med en offset på 0

60000/4096 = 14
14*4096 = 57344
60000-57344 = 2656
Adresse 60000 ligger i virtual page 14 med en offset på 2656

20000/8192 = 2
2*8192 = 16384
20000-16384 = 3616
Adresse 20000 ligger i virtual page 2 med en offset 3616

32768/8192 = 4
4*8192 = 32768
32768-32768 = 0
Adresse 32768 ligger i virtual page 4 med en offset på 0

60000/8192 = 7
7*8192 = 57344
60000-57344 = 2656
Adresse 60000 ligger i virtual page 7 med en offset på 2656

7. Tanenbaum oppgave 3.11
Suppose that a machine has 48-bit virtual addresses and 32-bit physical address.
(a) If pages are 4K, how many entries are in the page table if it has only a single level? Explain.
If there is only a single level we need to have all pages in our page table, this should be the entire 48 bit adresse spacce ( 2^48 ) divided by the page size ( 4K, 2^12 ). 
2^48 / 2^12 = 2^36 ( 64 GB ) 
(b) TLB with 32 entries. Suppose that a program contains instructions that fit into one page and it sequentially reads long integer elements from an array that spans thousands of pages. How effective will the TLB be for this case?

Litt utifra hvordan man skal tolke spørsmålet så kan dette use caset være veldig effektivt eller totalt ueffektivt. Hvis vi tenker at TLB`en blir spurt hver gang programmet trenger neste integer vil den kun bomme når du når neste page ( Gitt 4 bytes long integers og 4K pages så får vi 1000 ints per page )  og dette betyr at for hver tusende oppslag så bommer vi ( ca 99.9% effektiv ). Men hvis vi tenker at maskinen cacher hele pages før TLB`en og TLB bare blir spurt hver gang den trenger en ny page, så vil TLB´en alltid bomme. 

Lab exercises

1. Lag et script myprocinfo.bash som gir brukeren følgende meny med tilhørende funksjonalitet:
1 - Hvem er jeg og hva er navnet på dette scriptet?
2 - Hvor lenge er det siden siste boot?
3 - Hvor mange prosesser og tråder finnes?
4 - Hvor mange context switch'er fant sted siste sekund?
5 - Hvor stor andel av CPU-tiden ble benyttet i kernelmode
     og i usermode siste sekund?
6 - Hvor mange interrupts fant sted siste sekund?
9 - Avslutt dette scriptet

Velg en funksjon:

Løsningen ligger her : https://github.com/StianHB/IMT2282/blob/master/myprocinfo.bash


Chapter 8

Theory questions

1. Hva menes med page replacement? Beskriv hva som skjer ved denne aktiviteten.
Page replacement forekommer når vi trenger en page som ikke er i det fysiske minnet og det ikke er plass til pagen i det fysiske minnet. Man må da finne en page man kan fjerne ( skrive ut til disk ), skrive den til disk og så legge inn den pagen vi trenger i minnet. For å få til dette trengs det en metode for å finne den pagen man kan fjerne. Det finnes en teoretisk perfekt måte å gjøre dette på, men i praksis har man funnet gode løsninger som ligger opp mot den teoretiske løsningen.

2. Hva menes med working set og thrashing i forbindelse med minnehåndtering?
Working set er definert som mengden minne en prosess krever i ett gitt tidsintervall. ( http://en.wikipedia.org/w/index.php?title=Working_set&oldid=631454291 ) Tanenbaum definerer dette som "currently using", dvs at ett working set er mengden minne en prosses trenger akkurat nå. Thrashing er når en maskin/prossess ender opp med å måtte ut på disk for å hente pages så ofte at det går så mye ut over ytelsen til maskinen at den blir ubrukelig.

3. Hva er hensikten med et swapområde?
Et swapområde brukes til å flytte pages mellom RAM og disk.

4. A computer has 4 page frames. The time of loading, time of last access and the R(efernced) and M(odified) bits for each page are ( the times are in clock ticks):

| Page | Loaded | Last ref | R | M |
| 0    | 126    | 280      | 1 | 0 |
| 1    | 230    | 265      | 0 | 1 |
| 2    | 140    | 270      | 0 | 0 |
| 3    | 110    | 285      | 1 | 1 |

Which page will NRU replace?

NRU vil swappe page nr 2 til harddisk og plassere den nye pagen i den plassen. Dette er fordi page 2 ikke har blitt referert til siden sist klokke interrupt og den har heller ikke blitt modifisert. Dette plasserer page nr 2 i den laveste klassen etter NRU sin rangering.

Which page will FIFO replace?

FIFO er relativt rett fram og tar bare den første som kom inn, page 3 i dette tilfellet, og swapper den til disk.

Which page will LRU replace?

LRU swapper ut den page som det er lengst siden har blitt brukt ( referert til ) og i denne sammenhengen blir det page 1. 

Which page will second chance replace?
